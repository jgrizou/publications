%!TEX root = aaai2014.tex
Interactive learning deals with the problem of learning and solving tasks using human instructions. It is common in human-robot interaction, tutoring systems, and in human-computer interfaces such as brain-computer ones. In most cases, learning these tasks is possible because the signals are predefined or an ad-hoc calibration procedure allows to map signals to specific meanings. In this paper, we address the problem of simultaneously solving a task under human feedback and learning the associated meanings of the feedback signals. 
This has important practical application since the user can start controlling a device from scratch, without the need of an expert to define the meaning of signals or carrying out a calibration phase. The paper proposes an algorithm that simultaneously assign meanings to signals while solving a sequential task under the assumption that both, human and machine, share the same a priori on the possible instruction meanings and the possible tasks. Furthermore, we show using synthetic and real EEG data from a brain-computer interface that taking into account the uncertainty of the task and the signal is necessary for the machine to actively plan how to solve the task efficiently. 


% Recent work has shown that it is possible to extract feedback information from EEG measurements of brain activity, such as error potentials, and use it to solve sequential tasks. As most Brain-Computer Interfaces, a calibration phase is required to build a decoder that translates raw EEG signals to understandable feedback signals. This paper proposes a method to solve sequential tasks based on feedback extracted from the brain without any calibration. 

% We show that a signal decoder can be learnt automatically and online by the system under the assumption that both, human and machine, share the same a priori on the possible signals’ meanings and the possible tasks the user may want the device to achieve. We present computational results showing that: a) it is possible to learn the signal-to-meaning mapping of unlabelled and noisy teaching signals, as well as a new task at the same time, b) it is possible to reuse the acquired knowledge about teaching signals for learning new tasks and c) we improve over calibration based methods by automatically and online adaptation to the signals properties. We further introduce a planning strategy that exploits the task and signal uncertainty to allow more efficient learning sessions.

% Recent work has shown that it is possible to extract feedback information from EEG measurements of brain activity, such as error potentials, and use it to solve sequential tasks. As most Brain-Computer Interfaces, a calibration phase is required to build a decoder that translates raw EEG signals to understandable feedback signals. This paper proposes a method to solve sequential tasks based on feedback extracted from the brain without any calibration. 

% This paper argues and present empirical results showing that, under specific but realistic conditions, this problem can be solved. 

% where the user teaches a robot a new task using teaching instructions yet unknown to it. 

% In such cases, the robot needs to estimate simultaneously what the task is and the associated meaning of instructions received from the user. 

% We present an algorithm allowing a user to instruct a machine a new task using unlabelled instruction signals. For this work, we consider a scenarios where a human teacher uses signals whose associated meaning can be a feedback (correct/incorrect).

% We report online experiments where four users directly controlled an agent on a 2D grid world to reach a target without any previous calibration process.
